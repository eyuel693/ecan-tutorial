\documentclass[12pt,a4paper]{article}
\usepackage[utf8]{inputenc}

\usepackage{geometry}  % easy margin control (defaults are fine; optional)
\geometry{margin=1in}
\usepackage{setspace}
\setstretch{1.25}
\usepackage{graphicx}
\usepackage{booktabs}
\usepackage{xcolor}


% ----------- core mathematics -----------
\usepackage{amsmath}   % align, split, cases, etc.
\usepackage{amssymb}   % \mathbb, \mathcal, \leqslant, \geqslant ...
\usepackage{amsfonts}  % blackboard bold, fraktur if desired
\usepackage{graphicx}  % for future figures, even if none are included yet
\usepackage{listings}  % for code examples
\usepackage{hyperref}  % clickable cross-refs; load last
\usepackage{mdframed}  % for boxed content

% ----------- tables & arrays ------------
\usepackage{array}     % extended column specifiers in tabular
\usepackage{booktabs}  % nicer horizontal rules (optional; you may keep \hline)
\usepackage{multirow}  % multi-row cells if you extend the tables later

% ----------- layout & floats ------------

\usepackage{caption}   % better caption spacing for tables/figures
\documentclass{article}

% ---------- Required Packages ----------
\usepackage{tcolorbox}
\usepackage{listings}
\usepackage{enumitem}
\usepackage{noto}

\geometry{margin=1in}
\hypersetup{
    colorlinks=true,
    linkcolor=blue,
    urlcolor=cyan
}

% ---------- Define Colors ----------
\definecolor{lightblue}{rgb}{0.90, 0.95, 1.0}
\definecolor{deepblue}{rgb}{0.2, 0.4, 0.7}
\definecolor{codebg}{rgb}{0.96, 0.96, 0.96}

% ---------- MeTTa Code Environment ----------
\lstnewenvironment{metta}[1][]{
    \lstset{
        language=Lisp,
        basicstyle=\small\ttfamily,
        backgroundcolor=\color{codebg},
        frame=single,
        framesep=3mm,
        xleftmargin=5mm,
        xrightmargin=5mm,
        breaklines=true,
        showstringspaces=false,
        #1
    }
}{}
\title{Introduction }
\order{1}
\author{ECAN Team}
\date{October 12, 2025}

\begin{document}


\section{Introduction}

The \textbf{Economic Attention Network (ECAN)} The \textbf{Economic Attention Network (ECAN)} is a model for managing and distributing attention in cognitive systems. Every piece of knowledge in the system, called an \textbf{Atom}, has an \textbf{Attention Value}, which changes over time depending on its importance or relevance to the system’s current goals and activities.

In ECAN, attention dynamics are guided by principles of artificial economics. Two forms of ``currency'' are used:

\begin{itemize}
    \item \textbf{Short-Term Importance (STI)} -- captures immediate or temporary relevance.
    \item \textbf{Long-Term Importance (LTI)} -- represents more permanent or strategic relevance.
\end{itemize}

These values are updated through nonlinear dynamical equations. A key mechanism is \textbf{Hebbian Learning}, where connections (Hebbian Links) are formed between Atoms that are often used together. For example, if Atom A helps activate Atom B frequently, B may ``reward'' A with some attention currency, ensuring that useful knowledge gains more attention over time.

In simple terms, ECAN functions like activation-spreading in neural networks, where important or connected nodes light up more strongly.

The framework includes a proof-of-concept (POC) built in \textbf{MeTTa}, an agent-based cognitive environment. In this setup:

\begin{itemize}
    \item Multiple agents run simultaneously in a shared memory space called \textbf{Atomspace}.
    \item Agents can directly see and use each other’s knowledge.
    \item The system maintains its state as long as it runs, supporting interaction and collaboration between agents.
\end{itemize}

Together, these features make ECAN a powerful method for modeling how artificial minds can manage focus, prioritize knowledge, and coordinate learning. From ECAN, we build toward full components like Atomspace (knowledge storage) and MeTTa (programming language), culminating in end-to-end AGI applications.

\subsection{Why Attention Allocation is Essential}

Attention allocation is crucial for efficient cognitive systems, especially in AGI frameworks like Hyperon. Its benefits include:

\begin{enumerate}
    \item \textbf{Resource Optimization:}
    \item \textbf{Cognitive Focus (STI vs. LTI):} Mimics human attention by directing reasoning toward urgent, goal-relevant information. ECAN achieves this through dual currency:
    \begin{itemize}
        \item \textbf{STI:} Highlights immediate priorities, such as diagnosing a current fault.
        \item \textbf{LTI:} Retains valuable generalized knowledge for future use, preserving past fault patterns.
    \end{itemize}
    \item \textbf{Noise Handling \& Enhanced Inference:} Filters messy or misleading real-world data, enhancing inference accuracy. The \textbf{Attentional Focus (AF)} mechanism ensures only the most important atoms are considered in reasoning.
    \item \textbf{Scalability for Large Knowledge Graphs:} Dynamically creates a manageable Attentional Focus, preventing overload in large distributed knowledge bases like Hyperon’s Atomspace.
    \item \textbf{Synergy with Reasoning (Cognitive Synergy):} Guides probabilistic inference (PLN) toward relevant knowledge, ensuring correct application of rules for tasks like fault classification.
    \item \textbf{Real-Time Adaptability:} Updates dynamically based on stimuli, allowing the system to respond quickly to new inputs, such as sudden faults.
\end{enumerate}

\subsection{Tutorial Objectives}

By the end of this tutorial, learners will be able to:

\begin{enumerate}
  \item \textbf{Understand the Concept of ECAN}
      \begin{itemize}
          \item Gain a clear understanding of \textbf{Economic Attention Allocation (ECAN)} and its role in managing attention in cognitive systems.
          \item Learn how ECAN helps prioritize knowledge and focus processing resources effectively.
      \end{itemize}
  \item \textbf{Learn the Role of Mind Agents}
      \begin{itemize}
          \item Explain how \textbf{Mind Agents} manage, update, and utilize knowledge in the \textbf{AtomSpace}.
          \item Understand how these agents interact with attention values and Hebbian Links to support reasoning and learning.
      \end{itemize}
  \item \textbf{Understand Attention Metrics: STI and LTI}
      \begin{itemize}
          \item Describe \textbf{Short-Term Importance (STI)} and \textbf{Long-Term Importance (LTI)}.
          \item Learn how these metrics guide attention allocation to relevant atoms.
      \end{itemize}
  \item \textbf{Explore Hebbian Links}
      \begin{itemize}
          \item Understand the role of \textbf{Hebbian Links} in spreading attention between related atoms.
          \item Learn about different types of links: symmetric, asymmetric, and inverse, and their effects on attention distribution.
      \end{itemize}
  \item \textbf{Examine Key Agent Workflows}
      \begin{itemize}
          \item Explain the workflow of critical agents, including \textbf{ImportanceDiffusionAgent}, \textbf{ForgettingAgent}, and Hebbian Updating/Creation Agents.
          \item Understand how these agents collaborate to maintain a dynamic and adaptive cognitive system.
      \end{itemize}
  \item \textbf{Apply ECAN in AI Systems}
      \begin{itemize}
          \item Apply the principles of attention allocation to adaptive reasoning and knowledge management.
          \item Explore practical use cases of ECAN for AI tasks, such as fault detection and decision-making.
      \end{itemize}
\end{enumerate}

\end{document}