\documentclass[12pt,a4paper]{article}
\usepackage[utf8]{inputenc}

\usepackage{geometry}  % easy margin control (defaults are fine; optional)
\geometry{margin=1in}
\usepackage{setspace}
\setstretch{1.25}
\usepackage{graphicx}
\usepackage{booktabs}
\usepackage{xcolor}


% ----------- core mathematics -----------
\usepackage{amsmath}   % align, split, cases, etc.
\usepackage{amssymb}   % \mathbb, \mathcal, \leqslant, \geqslant ...
\usepackage{amsfonts}  % blackboard bold, fraktur if desired
\usepackage{graphicx}  % for future figures, even if none are included yet
\usepackage{listings}  % for code examples
\usepackage{hyperref}  % clickable cross-refs; load last
\usepackage{mdframed}  % for boxed content

% ----------- tables & arrays ------------
\usepackage{array}     % extended column specifiers in tabular
\usepackage{booktabs}  % nicer horizontal rules (optional; you may keep \hline)
\usepackage{multirow}  % multi-row cells if you extend the tables later

% ----------- layout & floats ------------

\usepackage{caption}   % better caption spacing for tables/figures
\documentclass{article}

% ---------- Required Packages ----------
\usepackage{tcolorbox}
\usepackage{listings}
\usepackage{enumitem}
\usepackage{noto}

\geometry{margin=1in}
\hypersetup{
    colorlinks=true,
    linkcolor=blue,
    urlcolor=cyan
}

% ---------- Define Colors ----------
\definecolor{lightblue}{rgb}{0.90, 0.95, 1.0}
\definecolor{deepblue}{rgb}{0.2, 0.4, 0.7}
\definecolor{codebg}{rgb}{0.96, 0.96, 0.96}

% ---------- MeTTa Code Environment ----------
\lstnewenvironment{metta}[1][]{
    \lstset{
        language=Lisp,
        basicstyle=\small\ttfamily,
        backgroundcolor=\color{codebg},
        frame=single,
        framesep=3mm,
        xleftmargin=5mm,
        xrightmargin=5mm,
        breaklines=true,
        showstringspaces=false,
        #1
    }
}{}

% ---------- Title ----------
\title{ How Attention Allocation Works in ECAN}
\order{6}
\author{ECAN Team}
\date{October 2025}

\begin{document}
\maketitle

\section{Overview}

This section explains the flow of attention allocation in the \textbf{Economic Attention Network (ECAN)}.  
It shows how different cognitive agents collaborate to maintain a focused, adaptive, and energy-efficient cognitive system.

Attention allocation in ECAN is a dynamic feedback process that continuously redistributes cognitive energy 
based on stimulus, memory persistence, and associative relevance.
\end{tcolorbox}

\section{Step-by-Step Process}

\subsection{1. Rewarding Useful Atoms}

Atoms receive stimulus from a \textbf{MindAgent} when they contribute to achieving goals.  
This stimulus is processed by the \textbf{ImportanceUpdatingAgent}, which converts it into two distinct forms of importance:

\begin{itemize}[leftmargin=*]
  \item \textbf{Short-Term Importance (STI)} — represents immediate activation or priority for processing.
  \item \textbf{Long-Term Importance (LTI)} — represents lasting knowledge or contextual importance.
\end{itemize}

\subsection{2. Spreading Importance}

Once STI is established, it spreads among connected atoms via \texttt{HebbianLinks}.  
This mechanism models associative learning, ensuring that related concepts are activated together.

Two primary agents handle this process:

\begin{itemize}[leftmargin=*]
  \item \textbf{ImportanceDiffusionAgent} — spreads STI probabilistically across linked atoms.
  \item \textbf{ImportanceSpreadingAgent} — performs deterministic spreading for structured attention distribution.
\end{itemize}

\subsection{3. Updating Hebbian Links}

The \textbf{HebbianUpdatingAgent} continuously modifies the truth value and strength of HebbianLinks 
based on co-activation within the \textbf{Attentional Focus (AF)}.  
If two atoms frequently activate together, their connection strengthens; if not, it weakens.


\emph{"Neurons that fire together, wire together."}  
In ECAN, the same principle governs atom connectivity active associations are reinforced dynamically through Hebbian updating.
\end{tcolorbox}

\subsection{4. Forgetting Mechanism}

To maintain memory efficiency, the \textbf{ForgettingAgent} manages decay and removal of less relevant atoms.  
This process prevents resource exhaustion and keeps the system adaptive.

\begin{itemize}[leftmargin=*]
  \item Atoms with LTI below a threshold are gradually pruned.
  \item Occasionally, higher-LTI atoms are removed to rebalance memory load.
\end{itemize}

\subsection{5. Creating New Links}

When new atoms enter the \textbf{Attentional Focus}, the \textbf{HebbianCreationModule} forms 
\texttt{AsymmetricHebbianLinks} between them and other active atoms.  
This mechanism expands the cognitive network, allowing the system to learn new associations dynamically.

\section{Conclusion}
Through coordinated operations of multiple agents \textbf{Updating}, \textbf{Diffusion}, \textbf{Forgetting}, 
and \textbf{Creation} ECAN maintains a self-balancing attention economy that prioritizes relevance, adaptability, 
and computational efficiency.


\end{document}
