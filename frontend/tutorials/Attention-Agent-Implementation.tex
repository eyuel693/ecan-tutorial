\documentclass[12pt,a4paper]{article}
\usepackage[utf8]{inputenc}

% ----------- layout & geometry -----------
\usepackage{geometry}
\geometry{margin=1in}
\usepackage{setspace}
\setstretch{1.25}

% ----------- core mathematics -----------
\usepackage{amsmath}   % align, split, cases, etc.
\usepackage{amssymb}   % \mathbb, \mathcal, \leqslant, \geqslant ...
\usepackage{amsfonts}  % blackboard bold, fraktur if desired

% ----------- graphics & colors -----------
\usepackage{graphicx}  % for figures
\usepackage{xcolor}    % for colors

% ----------- tables & arrays ------------
\usepackage{array}     % extended column specifiers in tabular
\usepackage{booktabs}  % nicer horizontal rules
\usepackage{multirow}  % multi-row cells

% ----------- code listings ------------
\usepackage{listings}  % for code examples

% ----------- boxes & frames -----------
\usepackage{tcolorbox}
\usepackage{mdframed}  % for boxed content

% ----------- lists & enum -----------
\usepackage{enumitem}  % for list customization

% ----------- fonts -----------
\usepackage{noto}      % Noto font

% ----------- hyperref (load last) -----------
\usepackage{hyperref}  % clickable cross-refs

% ---------- Define Colors ----------
\definecolor{lightblue}{rgb}{0.90, 0.95, 1.0}
\definecolor{deepblue}{rgb}{0.2, 0.4, 0.7}
\definecolor{codebg}{rgb}{0.96, 0.96, 0.96}

% ---------- MeTTa Code Environment ----------
\lstnewenvironment{metta}[1][]{
    \lstset{
        language=Lisp,
        basicstyle=\small\ttfamily,
        backgroundcolor=\color{codebg},
        frame=single,
        framesep=3mm,
        xleftmargin=5mm,
        xrightmargin=5mm,
        breaklines=true,
        showstringspaces=false,
        #1
    }
}{}

\title{Attention Agent Implementation}
\author{ECAN Team}
\order{5}
\date{October 12, 2025}

\begin{document}

\maketitle

\section{Attention Agent Implementation}

\subsection{Overview}
The Attention Agent subsystem dynamically manages the ECAN attention economy. It regulates how Short-Term Importance (STI) and Long-Term Importance (LTI) values evolve within the Atomspace.

Three main agent types maintain this economy:

\begin{itemize}[noitemsep]
    \item \textbf{Importance Diffusion Agents} – Spread STI between related atoms.
    \item \textbf{Rent Collection Agents} – Apply time-based decay to STI/LTI values.
    \item \textbf{Forgetting Agents} – Remove low-importance atoms to control memory size.
\end{itemize}

These ensure energy flow, decay, and selective forgetting within the cognitive system.

\subsection{Agent Architecture}
The Attention Agent framework models a cognitive economy where STI and LTI flow between atoms via associative relationships. Each agent type operates in a specific domain and is triggered differently.

\textbf{Agent Types:}

\begin{itemize}[noitemsep]
    \item \textbf{AF Diffusion:} Spreads STI within attentional focus (triggered each cycle).
    \item \textbf{WA Diffusion:} Spreads STI outside attentional focus (triggered each cycle).
    \item \textbf{AF Rent Collection:} Applies time-based decay (triggered by time).
    \item \textbf{Forgetting Agent:} Removes low-LTI atoms (triggered by memory size).
\end{itemize}

\textbf{Core Mechanisms:}

\begin{itemize}[noitemsep]
    \item Value conservation: STI is redistributed, not created.
    \item Natural decay: Attention fades over time.
    \item Memory bounds: Prevents overflow.
    \item Knowledge preservation: VLTI and Hebbian links retain critical data.
\end{itemize}

\subsection{1. Importance Diffusion Agents}
\textbf{Purpose:} Spread STI among connected atoms using link strengths and probabilities.

\textbf{Core Functions:}

\begin{itemize}[noitemsep]
    \item \texttt{diffuseAtom} – Orchestrates STI transfer.
    \item \texttt{incidentAtoms} – Finds connected non-Hebbian atoms.
    \item \texttt{hebbianAdjacentAtoms} – Identifies Hebbian-linked neighbors.
    \item \texttt{tradeSti} – Executes STI value exchange.
\end{itemize}

\textbf{Process:}

\begin{enumerate}[noitemsep]
    \item Identify connected and Hebbian neighbors.
    \item Compute transfer probabilities.
    \item Combine distributions.
    \item Execute redistribution via \texttt{tradeSti}.
\end{enumerate}

\textbf{AF Diffusion Agent:}
Operates within the attentional focus.  
- Source atoms chosen from AF set.  
- Transfers a fraction of STI based on a max spread percentage.  
- Target: Active working memory.

\textbf{WA Diffusion Agent:}
Operates outside AF.  
- Selects random non-AF atom.  
- Diffuses based on predefined amount.  
- Processes one atom per cycle.

\subsection{2. Rent Collection Agents}
\textbf{Purpose:} Simulate attention decay over time by reducing STI/LTI.

Implemented in \texttt{AFRentCollectionAgent.metta}. Tracks elapsed time and applies decay proportional to duration.

\textbf{Formulas:}
\[
\text{STI Rent} = (\text{calculateStiRent}) \times w
\]
\[
\text{LTI Rent} = (\text{calculateLtiRent}) \times w
\]

Minimum rent functions prevent negative STI or LTI values.

\textbf{Notes:}

\begin{itemize}[noitemsep]
    \item Prevents underflow (values never go below zero).
    \item Models gradual attention fading.
\end{itemize}

\subsection{3. Forgetting Agent}
\textbf{Purpose:} Manage memory size by deleting low-importance atoms.

Triggered when:
\[
\text{Size} > (\text{maxSize} + \text{accDivSize})
\]

\textbf{Core Steps:}

\begin{enumerate}[noitemsep]
    \item Filter atoms with LTI below threshold and VLTI = 0.
    \item Sort by LTI and truth value.
    \item Validate before deletion.
    \item Remove safely and refund STI/LTI to global pool.
\end{enumerate}

\textbf{Safety Rules:}

\begin{itemize}[noitemsep]
    \item Preserve atoms with active links.
    \item Refund deleted atoms' importance values.
    \item Ensure no dangling references.
\end{itemize}

\subsection{Agent Integration}
All agents operate cyclically to maintain balance:

\begin{enumerate}[noitemsep]
    \item Diffusion redistributes attention.
    \item Rent collection applies decay.
    \item Forgetting cleans obsolete atoms.
\end{enumerate}

Together, they maintain adaptive, stable attention within ECAN.

\end{document}